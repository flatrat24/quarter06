\documentclass[12pt]{article}

%%--- GRAPHICS ---%%
\usepackage{tikz}
\usepackage[siunitx, american, RPvoltages]{circuitikz}
\usetikzlibrary{arrows.meta}
\usepackage{tikz-3dplot}
\usepackage{graphicx}
\usepackage{pgfplots}
  \pgfplotsset{compat=1.18}
\usetikzlibrary{arrows}
\newcommand{\midarrow}{\tikz \draw[-triangle 90] (0,0) -- +(.1,0);}
\tikzset{
  every pin/.style={draw=fg,fill=fg!10!bg,rectangle,rounded corners=3pt},
  every pin edge/.style={thick, fg, <-},
  >=stealth
}

%%--- Font Options ---%%
\usepackage{inconsolata} % Monospace font

%%--- Code Formatting ---%%
\usepackage{listings}
\lstdefinestyle{catppuccin}{
  backgroundcolor   = \color{fg!5!bg},  % colour for the background. External color or xcolor package needed.
  commentstyle      = \color{gr},       % style of comments in source language
  keywordstyle      = \color{ma},       % style of keywords in source language (e.g. keywordstyle=\color{red})
  numberstyle       = \color{fg!50!bg}, % style used for line-numbers
  numbersep         = 5pt,              % distance of line-numbers from the code
  stringstyle       = \color{or},       % style of strings in source language
  basicstyle        = \ttfamily\small,  % font size, font family, etc
  breakatwhitespace = false,            % sets if automatic breaks should only happen at whitespaces
  breaklines        = true,             % automatic line-breaking
  captionpos        = b,                % position of caption (t/b)
  keepspaces        = true,             % keep spaces in the code, useful for indetation
  numbers           = left,             % position of line numbers (left/right/none, i.e. no line numbers)
  showspaces        = false,            % emphasize spaces in code (true/false)
  showstringspaces  = false,            % emphasize spaces in strings (true/false)
  showtabs          = false,            % emphasize tabulators in code (true/false)
  tabsize           = 2,                % default tabsize
  frame             = leftline,         % showing frame outside code (none/leftline/topline/bottomline/lines/single/shadowbox)
  rulecolor         = \color{fg}        % Specify the colour of the frame-box
}
\lstset{style=catppuccin}

%%--- FIGURES ---%%
\usepackage{subcaption}
\usepackage{wrapfig}
\usepackage{float}
\usepackage[skip=5pt, font=footnotesize]{caption}

%%--- FORMATTING ---%%
\usepackage{parskip}
\usepackage{tcolorbox}
\usepackage{ulem}

%%--- TABLE FORMATTING ---%%
\usepackage{tabularray}
\UseTblrLibrary{booktabs}

%%--- MATH AND LOGIC ---%%
\usepackage{xifthen}
\usepackage{amsmath}
\usepackage{amssymb}
\usepackage{amsfonts}

%%--- TEXT AND SYMBOLS ---%%
\usepackage[T1]{fontenc}
\usepackage{textcomp}
\usepackage{gensymb}

%%--- OTHER ---%%
\usepackage{standalone}

%%--- LOGIC SYMBOLS ---%%
\newcommand*\xor{\oplus}

%%--- STYLES ---%%

% Packages
\usepackage[paper=letterpaper,tmargin=45pt,bmargin=45pt,lmargin=45pt,rmargin=45pt]{geometry}
\usepackage{titlesec}
\usepackage[rgb]{xcolor}
\selectcolormodel{natural}
\usepackage{ninecolors}
\selectcolormodel{rgb}

% Colors
\definecolor{pg}{HTML}{24273A}
\definecolor{fg}{HTML}{FFFFFF}
\definecolor{bg}{HTML}{24273A}
\definecolor{re}{HTML}{F38BA8}
\definecolor{gr}{HTML}{A6E3A1}
\definecolor{ye}{HTML}{F9E2AF}
\definecolor{or}{HTML}{FAB387}
\definecolor{bl}{HTML}{89B4FA}
\definecolor{ma}{HTML}{CBA6F7}
\definecolor{cy}{HTML}{94E2D5}
\definecolor{pi}{HTML}{F2CDCD}

\definecolor{copper}{HTML}{B87333}

\usepackage{nameref}
\makeatletter
\newcommand*{\currentname}{\@currentlabelname}
\makeatother

\titleformat{\section}
{\normalfont\scshape\Large\bfseries}
{\thesection}
{0.75em}
{}

\titleformat{\subsection}
{\normalfont\scshape\large\bfseries}
{\thesubsection}
{0.75em}
{}

\titleformat{\subsubsection}
{\normalfont\scshape\normalsize\bfseries}
{\thesubsubsection}
{0.75em}
{}

% Formula
\newcounter{formula}[section]
\newenvironment{formula}[1]{
  \stepcounter{formula}
  \begin{tcolorbox}[
    standard jigsaw, % Allows opacity
    colframe={fg},
    boxrule=1px,
    colback=bg,
    opacityback=0,
    sharp corners,
    sidebyside,
    righthand width=25px,
    coltext={fg}
    ]
    \centering
    \textbf{\uline{#1}}
  }{
    \tcblower
    \textbf{\thesection.\theformula}
  \end{tcolorbox}
}

% Definition
\newcounter{definition}[section]

\newenvironment{definition*}[1]{
  \begin{tcolorbox}[
    standard jigsaw, % Allows opacity
    colframe={fg},
    boxrule=1px,
    colback=bg,
    opacityback=0,
    sharp corners,
    coltext={fg}
    ]
    \textbf{#1 \hfill}
    \vspace{5px}
    \hrule
    \vspace{5px}
    \noindent
  }{
  \end{tcolorbox}
}

\newenvironment{definition}[1]{
  \stepcounter{definition}
  \begin{tcolorbox}[
    standard jigsaw, % Allows opacity
    colframe={fg},
    boxrule=1px,
    colback=bg,
    opacityback=0,
    sharp corners,
    coltext={fg}
    ]
    \textbf{#1 \hfill \thesection.\thedefinition}
    \vspace{5px}
    \hrule
    \vspace{5px}
    \noindent
  }{
  \end{tcolorbox}
}

% Example Problem
\newcounter{example}[section]
\newenvironment{example}[1]{
  \stepcounter{example}
  \begin{tcolorbox}[
    standard jigsaw, % Allows opacity
    colframe={fg},
    boxrule=1px,
    colback=bg,
    opacityback=0,
    sharp corners,
    coltext={fg}
    ]
    \textbf{Example - #1 \hfill \thesection.\theexample}
    \vspace{5px}
    \hrule
    \vspace{5px}
    \noindent
  }{
  \end{tcolorbox}
}

\tikzset{
  % Electrical/Computer Engineering
  wire/.style = {
    draw = fg,
    thick
  },
  % Mechanical Engineering
  body/.style = {
    fill = bl!25!bg,
    draw = fg,
    thick,
    fill opacity = 0.95
  },
  force/.style = {
    draw = re,
    ultra thick,
    -stealth
  },
  dimension/.style = {
    draw = fg,
    thick,
    |-|
  },
  support/.style = {
    fill = brown!50!bg,
    draw = brown!50!bg,
    ultra thick
  },
  line/.style = {
    draw = fg,
    thick
  }
}

\pgfplotsset{
  axis/.style={
    grid,
    major grid style={line width=.2pt,draw=fg!50!bg},
    axis lines = box,
    axis line style = {line width = 1px}
  }
}

%%--- REFERENCES ---%%
\usepackage{hyperref}
\hypersetup{
  colorlinks  = true,
  linkcolor   = bl,
  anchorcolor = bl,
  citecolor   = bl,
  filecolor   = bl,
  menucolor   = bl,
  runcolor    = bl,
  urlcolor    = bl,
}

\author{Ethan Anthony}
\newcommand*{\equal}{=}


\title{Lecture 002}
\date{April 06, 2025}

\begin{document}

\newpage
\section{First Order Circuits}
\label{sec:firstOrderCircuits}

Now having the tools to analyze capacitors and inductors, circuits beyond simple resistance-based ones can be analyzed. Previously with circuit consisting of only sources and resistors, applying Kirchoff's Laws resulted in algebraic expressions that represented the circuit.

With capacitors and inductors, applying these same laws will yield differential equations that represent the circuit. Considering that the equation for the current across a capacitor is:
\begin{equation*}
  i = C\frac{dv}{dt}
\end{equation*}
and the voltage across an inductor is:
\begin{equation*}
  v = L\frac{di}{dt}
\end{equation*}
any set of equations involving one of these equalities would contain derivatives of the first order at most. Thus, circuits involving a single capacitor or inductor are \textbf{First Order Circuits}.

\begin{definition}{First Order Circuit}
  A First Order Circuit is a circuit that is characterized by a first order differential equation.
\end{definition}

\subsection{RC Circuits}
\label{ssec:rcCircuits}

\begin{wrapfigure}[4]{r}{0.2\textwidth}
  \centering
  \includestandalone{figures/fig_024}
  \caption{RC Circuit}
  \label{fig:024}
\end{wrapfigure}

An \textbf{RC Circuit} is a foundational circuit both for its simplicity as well as how widely it is used in practice. The basic configuration of the circuit is seen in Figure \ref{fig:024}.

\begin{definition}{RC Circuit}
  An RC Circuit is a circuit comprising of a resistor (R) and a capacitor (C).
\end{definition}

\subsubsection{RC Circuit Without External Source}
\label{sssec:rcCircuitWithoutExternalSource}

Consider the circuit in Figure \ref{fig:025} where there is no independent voltage source connected in the circuit. There is only a capacitor and a resistor connected in series. Additionally, assume that the capacitor is initially charged (which is where the energy in the circuit will come from).

\begin{figure}[H]
  \centering
  \includestandalone{figures/fig_025}
  \caption{Source Free RC Circuit}
  \label{fig:025}
\end{figure}

Applying KCL at the ground node of the circuit, it can be seen that:
\begin{equation*}
  {\color{re} i_C} + {\color{gr} i_R} = 0
\end{equation*}
By definition:
\begin{equation*}
  {\color{re} i_C = C\frac{dv}{dt}}\ \ \ \textup{and}\ \ \ {\color{gr} i_R = \frac{v}{R}}
\end{equation*}
Thus, the KCL equation can be expanded to:
\begin{equation*}
  \left[{\color{re} C\frac{dv}{dt}} + {\color{gr} \frac{v}{R}} = 0\right] \rightarrow \left[ \frac{dv}{dt} + \frac{v}{CR} = 0 \right]
\end{equation*}
Here it should be apparent that this is a first order differential equation. To solve it, first rearrange it:
\begin{equation*}
  \left[ \frac{dv}{dt} + \frac{v}{CR} = 0 \right] \rightarrow \left[ \frac{dv}{{\color{or} dt}} = -\frac{{\color{ma} v}}{CR} \right] \rightarrow \left[ {\color{ma} \frac{1}{v}} dv = -\frac{1}{CR} {\color{or} dt} \right]
\end{equation*}
Then, integrating both sides:
\begin{equation*}
  \left[ {\color{ma} \frac{1}{v}} dv = -\frac{1}{CR} {\color{or} dt} \right] \rightarrow \int \rightarrow \left[ \ln|v| = -\frac{1}{CR} \cdot t + C \right]
\end{equation*}
Imagine the integration constant $C$ as being $\ln|A|$. In other words, set $C = \ln|A|$:
\begin{multline*}
  \left[ \ln|v| = -\frac{1}{CR} \cdot t + {\color{gr} C} \right] \rightarrow \left[ \ln|v| = -\frac{1}{CR} \cdot t + {\color{gr} \ln|A|} \right] \\ \rightarrow \left[ \ln|v| - {\color{gr} \ln|A|} = -\frac{1}{CR} \cdot t \right] \rightarrow \left[ \ln\left|\frac{v}{A}\right| = -\frac{1}{CR}\cdot t \right]
\end{multline*}
Then, exponentiating both sides:
\begin{equation*}
  \left[ {\color{ma} \ln\left|\frac{v}{A}\right|} = {\color{or} -\frac{1}{CR}\cdot t} \right] \rightarrow e^x \rightarrow \left[ e^{{\color{ma} \ln\left|\frac{v}{A}\right|}} = e^{{\color{or} \frac{-t}{CR}}} \right] \rightarrow \left[ \frac{v}{{\color{re} A}} = e^{\frac{-t}{CR}} \right] \rightarrow \left[ v = {\color{re} A}e^{\frac{-t}{CR}} \right]
\end{equation*}
Considering that the capacitor in the circuit was initially charged, how will that initial charge be represented in the characteristic equation? Recall that $A$ was the integration constant, so based on the initial conditions: $A = v(0) = V_i$. Thus:
\begin{equation*}
  \left[ v = Ae^{\frac{-t}{CR}} \right] \rightarrow \left[ v = V_{i}e^{\frac{-t}{CR}} \right]
\end{equation*}
This equations is the characteristic equation of a simple, source-free, RC Circuit. Notice the behavior of the equation; starting at $t=0$, the voltage in the circuit is initially $V_i$ and decays exponentially over time.
\begin{formula}{Natural Response of an RC Circuit}
  \begin{equation*}
    v = V_ie^{-\frac{t}{CR}} = V_ie^{-\frac{t}{\tau}}
  \end{equation*}
\end{formula}
The rate at which the voltage decay is important. It's known that the decay is exponential, but the decay can be described more specifically than just "exponentially". For some exponential function:
\begin{equation*}
  e^{\frac{t}{\tau}}
\end{equation*}
the function will decay by a factor of $\frac{1}{e}$ every $\tau$ seconds (assuming $t$ is measured in seconds). This value $\tau$ is called the \textbf{time constant}. So, with the function for the RC Circuit:
\begin{equation*}
  V_ie^{-\frac{t}{CR}}
\end{equation*}
it can be seen that $\tau = CR$.

The voltage of the circuit can be graphed over time to visually see the behavior of the circuit.

\begin{figure}[H]
  \centering
  \includestandalone{figures/fig_026}
  \caption{Voltage Response of the RC Circuit}
  \label{fig:026}
\end{figure}

Since all the behavior of the circuit discussed so far is the behavior of the circuit without external interference (some voltage or current source), this is considered the \textbf{natural response} of the circuit.
\begin{definition}{Natural Response}
  The natural response of a circuit refers to the behavior (in terms of voltages and currents) of the circuit itself, with no external sources of excitation.
\end{definition}

\subsubsection{Power and Energy}
\label{sssec:sfrccPowerAndEnergy}

Considering the circuit from the perspective of the resistor, by applying Ohm's Law to the voltage function, a similar function to model the current through the circuit can also be derived:
\begin{equation*}
  i_R(t) = \frac{v(t)}{R} = \frac{V_i}{R}e^{-\frac{t}{RC}}
\end{equation*}
Knowing the $p=vi$, the power dissipated by the resistor is:
\begin{equation*}
  p = {\color{re} v} {\color{gr} i_R} = {\color{re} V_ie^{-\frac{t}{RC}}} \cdot {\color{gr} \frac{V_i}{R}e^{-\frac{t}{RC}}} = \frac{V_i^2}{R}e^{-\frac{2t}{RC}}
\end{equation*}
and integrating the power dissipated gives the total energy absorbed by the resistor:
\begin{equation*}
  w = \frac{V_i^2}{R} \int_{0}^{t}e^{-\frac{2 \lambda}{RC}} \,d \lambda = -\frac{RCV_i^2}{2R}\left[ e^{-\frac{2 \lambda}{RC}} \right]_{0}^{t} = -\frac{CV_i^2}{2}\left(e^{-\frac{2(0)}{RC}}-e^{-\frac{2(t)}{RC}}\right) = -\frac{CV_i^2}{2}\left(1-e^{-\frac{2t}{RC}}\right)
\end{equation*}
Notice that as $t \rightarrow \infty$, $w_R \rightarrow \frac{1}{2}CV_i^2$ which is the same as the initial energy stored in the capacitor. This drives home the point that, throughout the life of the circuit, the energy initially stored in the capacitor is being dissipated by the resistor.

\begin{formula}{Energy in a Capacitor}
  \begin{equation*}
    w_C = \frac{1}{2}CV_i^2
  \end{equation*}
\end{formula}

\subsubsection{RC Circuit with External Voltage Source}
\label{sssec:rcCircuitWithExternalSource}

In many cases, an RC Circuit won't exist without any external source supplying voltage to it. When an external source is applied at some time, the source can be modeled as a \textbf{step function}.

\begin{definition}{Step Function}
  A step function is a function that is low (zero) by default until some time $\alpha$ where it jumps to some high value. The \textit{unit} step function starts at zero and goes to one. Expressed with notation, it is:
  \begin{equation*}
    \mathcal{U}(t-\alpha) = \begin{cases}
      0 ,&\ t < \alpha \\
      1 ,&\ \alpha \leq 0
    \end{cases}
  \end{equation*}
\end{definition}

\begin{wrapfigure}[]{r}{0.5\textwidth}
  \vspace{-15pt}
  \centering
  \begin{subfigure}[b]{0.23\textwidth}
    \centering
    \includestandalone{figures/fig_028}
    \caption{$t<0$}
    \label{fig:028}
  \end{subfigure}
  \begin{subfigure}[b]{0.23\textwidth}
    \centering
    \includestandalone{figures/fig_029}
    \caption{$t\geq0$}
    \label{fig:029}
  \end{subfigure}
  \caption{RC Circuit with a Step Input}
  \label{fig:rcCircuitWithAStepInput}
\end{wrapfigure}

Consider the circuit in Figure \ref{fig:rcCircuitWithAStepInput}. Applying KCL to the circuit, the following equation can be found:
\begin{equation*}
  {\color{re} i_C} + {\color{gr} i_R} = 0
\end{equation*}
or, through Ohm's law, it can be rewritten as:
\begin{equation*}
  \left[{\color{re} C\frac{dv}{dt}} + {\color{gr} \frac{v_C - V_s\mathcal{U}(t)}{R}} = 0\right] \rightarrow \left[\frac{dv}{dt} + \frac{v}{RC} = \frac{V_s \mathcal{U}(t)}{RC}\right]
\end{equation*}
Considering only time after the switch has been closed, $\mathcal{U}(t)=1$:
\begin{equation*}
  \left[\frac{dv}{dt} + \frac{v}{RC} = \frac{{\color{ma} V_s \mathcal{U}(t)}}{RC}\right] = \left[\frac{dv}{dt} + \frac{v}{RC} = \frac{{\color{ma} V_s}}{RC}\right]
\end{equation*}
then, further rearranging the equation and integrating:
\begin{equation*}
  \left[\frac{{\color{gr} dv}}{{\color{or} dt}} + \frac{{\color{bl} v}}{{\color{re} RC}} = \frac{{\color{bl} V_s}}{{\color{re} RC}}\right] \rightarrow \left[\frac{1}{{\color{bl} v-V_s}} {\color{gr} dv} = -\frac{1}{{\color{re} RC}} {\color{or} dt}\right] \rightarrow \int \rightarrow \left[\ln\left|v-V_s\right|=-\frac{t}{RC}\right]
\end{equation*}
What bounds are being integrated over? For $dt$, it would simply be from $t=0$ to $t=t$. For $dv$, the initial condition is whatever charge the capacitor had \textit{at} $t=0$: $v(0)=V_0$. The final condition is the charge of the capacitor at some time $t$: $v(t)$. Thus:
\begin{equation*}
  \Big[\ln\left|v-V_s\right|\Big]_{{\color{re} V_0}}^{{\color{gr} v(t)}} = \left[-\frac{t}{RC}\right]_{{\color{bl} 0}}^{{\color{or} t}}
\end{equation*}
which can be rearranged as:
\begin{multline*}
  \left[\ln|{\color{gr} v(t)}-V_s| - \ln|{\color{re} V_0}-V_s| = -\frac{{\color{or} t}}{RC} + \frac{{\color{bl} 0}}{RC}\right] \rightarrow \left[\ln\left|\frac{v(t)-V_s}{V_0-V_s}\right| = -\frac{t}{RC}\right] \\ \rightarrow e^x \rightarrow \left[\frac{v(t)-V_s}{{\color{pi} V_0-V_s}} = e^{-\frac{t}{RC}}\right] \rightarrow \left[v(t)-{\color{ma} V_s} = ({\color{pi} V_0-V_s})e^{-\frac{t}{RC}}\right] \rightarrow \left[v(t) = {\color{ma} V_s} + (V_0-V_s)e^{-\frac{t}{RC}}\right]
\end{multline*}
Recall that this is all during the interval of $t\geq0$, while before $t=0$, the capacitor had some arbitrary charge of $V_0$. Thus:
\begin{equation*}
  v(t) = \begin{cases}
    V_0 ,&\ t < 0 \\
    V_s + (V_0-V_s)e^{-\frac{t}{RC}} ,&\ 0 \leq t \\
  \end{cases}
\end{equation*}
This equation models the \textbf{complete response} of the RC Circuit.
 
\begin{formula}{Complete Response of an RL Circuit}
  \begin{align*}
    v(t) &= v(\infty) + \big[v(0)-v(\infty)\big]e^{-\frac{t}{\tau}} \\
         &= V_ie^{-\frac{t}{\tau}} + V_s\left(1-e^{-\frac{t}{\tau}}\right) \\
         &= \begin{cases}
           V_0 ,&\ t < 0 \\
           V_s + (V_0-V_s)e^{-\frac{t}{\tau}} ,&\ 0 \leq t \\
         \end{cases}
  \end{align*}
\end{formula}

This is because it incorporates both the \textit{natural} and \textit{forced} responses of the circuit. The natural response, as described earlier, is the behavior of the circuit due to the initial charge in the capacitor. The forced response is the behavior due to the external source applied to the circuit.

\begin{definition}{Forced Response}
  The forced response of a circuit refers to the behavior of the circuit as caused by the application of some external voltage or current source.
\end{definition}

A more conceptual way to view the equation is in terms of the natural response ($v_n$) and forced response ($v_f$) directly:
\begin{equation*}
  v(t) = v_n + v_f
\end{equation*}
where:
\begin{equation*}
  v_n = V_0e^{-\frac{t}{RC}};\ v_f = V_s\left(1-e^{-\frac{t}{RC}}\right)
\end{equation*}
Furthermore, a practical way to view the equation is:
\begin{equation*}
  v(t) = v(\infty) + \left[ v(0) - v(\infty) \right]e^{-\frac{t}{RC}}
\end{equation*}
where:
\begin{equation*}
  v(\infty) = \textup{final steady-state voltage}; v(0) = \textup{initial voltage}
\end{equation*}
which makes solving the circuit as straightforward as finding:
\begin{enumerate}
  \itemsep0em
  \item $v(0)$: the initial voltage across the capacitor
  \item $v(\infty)$: the final voltage across the capacitor
  \item $\tau=RC$: the time constant of the circuit
\end{enumerate}

\subsection{RL Circuits}
\label{ssec:rlCircuits}

An \textbf{RL Circuit} occupies a similar space in terms of its simplicity and popularity as the RC Circuit does. Functionally, the only difference is that the RL Circuit uses an inductor where the RC Circuit uses a capacitor.

\begin{definition}{RL Circuit}
  An RL Circuit is a circuit comprising of a resistor (R) and an inductor (L).
\end{definition}

\subsubsection{RL Circuit Without External Source}
\label{sssec:rlCircuitWithoutExternalSource}

Consider the circuit in Figure \ref{fig:027}. This circuit is a source free RL circuit seeing as there is an inductor in series with a resistor (RL circuit) and no voltage/current sources (source free).

\begin{figure}[H]
  \centering
  \includestandalone{figures/fig_027}
  \caption{Source Free RL Circuit}
  \label{fig:027}
\end{figure}

Applying KVL on the circuit, the following equation can be derived:
\begin{equation*}
  {\color{re} v_L} + {\color{gr} v_R} = 0
\end{equation*}
By definition:
\begin{equation*}
  {\color{re} v_L = L\frac{di}{dt}}\ \ \ \textup{and}\ \ \ {\color{gr} v_R = iR}
\end{equation*}
Thus, the KVL equation can be rewritten as:
\begin{equation*}
  \left[{\color{re} L\frac{di}{dt}} + {\color{gr} iR} = 0\right] \rightarrow \left[ \frac{di}{dt} + \frac{iR}{L} = 0 \right]
\end{equation*}
Just like the previous RC Circuit, the basic RL Circuit is seen to be a first order circuit characterized by its first order differential equation.
\begin{equation*}
  \left[ \frac{di}{dt} + \frac{iR}{L} = 0 \right] \rightarrow \left[ \frac{di}{dt} = - \frac{iR}{L} \right] \rightarrow \left[ \frac{1}{i}di = - \frac{R}{L} dt \right]
\end{equation*}
Then, integrating both sides:
\begin{equation*}
  \left[ \frac{1}{i}di = - \frac{R}{L} dt \right] \rightarrow \int \rightarrow \left[ \ln|i| = -\frac{R}{L} t + C \right]
\end{equation*}
Again, setting $C = \ln|A|$:
\begin{equation*}
  \left[ \ln|i| = -\frac{R}{L} t + C \right] \rightarrow \left[ \ln|i| = -\frac{R}{L} t + \ln|A| \right] \rightarrow \left[ \ln|i| - \ln|A| = -\frac{R}{L} t \right] \rightarrow \left[ \ln\left|\frac{i}{A}\right| = -\frac{R}{L} t \right]
\end{equation*}
Then, exponentiating both sides:
\begin{equation*}
  \left[ \ln\left|\frac{i}{A}\right| = -\frac{R}{L} t \right] \rightarrow e^x \rightarrow \left[ e^{\ln\left|\frac{i}{A}\right|} = e^{-\frac{R}{L} t} \right] \rightarrow \left[ \frac{i}{A} = e^{-\frac{R}{L} t} \right] \rightarrow \left[ i = Ae^{-\frac{R}{L} t} \right]
\end{equation*}
Again, $A$ represents the initial conditions of the circuit. Previously $A=V_i$ to account for the initial charge built up in the capacitor. Now with the inductor, $A=I_i$ for the initial current in the inductor.
\begin{equation*}
  \left[ i = Ae^{-\frac{R}{L} t} \right] \rightarrow \left[ i = I_ie^{-\frac{R}{L} t} \right]
\end{equation*}
This now gives the characteristic equation for a source free RL circuit.
\begin{formula}{Characteristic Equation of a RL Circuit}
  \begin{equation*}
    i = I_ie^{-\frac{R}{L}t} = I_ie^{-\frac{t}{\tau}}
  \end{equation*}
\end{formula}
This shows that the RL Circuit still experiences exponential decay after starting with some initial condition ($i=I_i$). However, now the time constant is different, being:
\begin{equation*}
  \tau = \frac{L}{R}
\end{equation*}

\subsubsection{Power and Energy}
\label{sssec:sfrlcPowerAndEnergy}

What would the voltage drop across the resistor be? By applying Ohm's Law:
\begin{equation*}
  v_R = iR = I_ie^{-\frac{t}{\tau}} \cdot R
\end{equation*}
Knowing the $p=vi$, the power dissipated by the resistor is:
\begin{equation*}
  p = {\color{re} v} {\color{gr} i_R} = {\color{re} I_iRe^{-\frac{t}{\tau}}} \cdot {\color{gr} I_ie^{-\frac{t}{\tau}}} = I_i^2Re^{-\frac{2t}{\tau}}
\end{equation*}
then integrating to find the total energy absorbed by the resistor:
\begin{equation*}
  w = I_i^2R \int_{0}^{t} e^{-\frac{2 \lambda}{\tau}} \,d \lambda = - \frac{\tau}{2} I_i^2R \left[ e^{-\frac{2 \lambda}{\tau}} \right]_{0}^{t} = - \frac{\tau}{2} I_i^2R \left( e^{-\frac{2(0)}{\tau}}-e^{-\frac{2(t)}{\tau}} \right) = - \frac{\tau}{2} I_i^2R \left( 1-e^{-\frac{2(t)}{\tau}} \right)
\end{equation*}
Notice that as $t \rightarrow \infty$, $w_R \rightarrow \frac{1}{2}LI_i^2$ which is the same as the initial energy stored in the inductor. Again, this shows the behavior of the resistor dissipating over time the energy initially stored in the inductor.

\begin{formula}{Energy in a Inductor}
  \begin{equation*}
    w_L = \frac{1}{2}LI_i^2
  \end{equation*}
\end{formula}

\subsubsection{RL Circuit with External Voltage Source}
\label{sssec:rlCircuitWithExternalSource}

\begin{wrapfigure}[12]{r}{0.5\textwidth}
  \vspace{-15pt}
  \centering
  \begin{subfigure}[b]{0.2\textwidth}
    \centering
    \includestandalone{figures/fig_030}
    \caption{$t<0$}
    \label{fig:030}
  \end{subfigure}
  \begin{subfigure}[b]{0.2\textwidth}
    \centering
    \includestandalone{figures/fig_031}
    \caption{$t\geq0$}
    \label{fig:031}
  \end{subfigure}
  \caption{RL Circuit with a Step Input}
  \label{fig:rlCircuitWithAStepInput}
\end{wrapfigure}

Consider the circuit in Figure \ref{fig:rlCircuitWithAStepInput}. A process similar to the process detailed in Subsubsection \ref{sssec:rcCircuitWithExternalSource} can be done, starting with KVL applied to the circuit. However, rather than doing that, consider the equation:
\begin{equation*}
  i = {\color{ma} i_n} + {\color{or} i_f}
\end{equation*}
as determined in Subsubsection \ref{sssec:rlCircuitWithoutExternalSource}, the natural response ({\color{ma} $i_n$}) of an RL circuit is generally a decaying exponential:
\begin{equation*}
  {\color{ma} i_n} = Ce^{-\frac{t}{{\color{gr} \tau}}};\ {\color{gr} \tau} = \frac{L}{R}
\end{equation*}
and the final behavior of the circuit, when the inductor becomes a short circuit, will be:
\begin{equation*}
  {\color{or} i_f} = \frac{V_s}{R}
\end{equation*}
substituting back into the first equation:
\begin{equation*}
  i = {\color{ma} Ce^{-\frac{t}{\tau}}} + {\color{or} \frac{V_s}{R}}
\end{equation*}
What is the constant $C$? At $t=0$, the current equation for $i$ becomes:
\begin{equation*}
  \left[{\color{re} i} = Ce^{-\frac{{\color{gr} t}}{\tau}} + \frac{V_s}{R}\right] \rightarrow \left[{\color{re} I_0} = Ce^{-\frac{{\color{gr} 0}}{\tau}} + \frac{V_s}{R}\right] \rightarrow \left[{\color{re} I_0} = C + \frac{V_s}{R}\right] \rightarrow \left[C = I_0 - \frac{V_s}{R}\right]
\end{equation*}
\clearpage
Then, substituting back into the equation:
\begin{equation*}
  \left[i = {\color{ma} C}e^{-\frac{t}{\tau}} + \frac{V_s}{R}\right] \rightarrow \left[i = \left({\color{ma} I_0-\frac{V_s}{R}}\right)e^{-\frac{t}{\tau}} + \frac{V_s}{R}\right]
\end{equation*}
Which now gives the \textbf{complete response} of the RL Circuit:
\begin{formula}{Complete Response of an RL Circuit}
  \begin{align*}
    i(t) &= i(\infty) + \big[i(0)-i(\infty)\big]e^{-\frac{t}{\tau}} \\
         &= I_ie^{-\frac{t}{\tau}} + I_s\left(1-e^{-\frac{t}{\tau}}\right) \\
         &= \begin{cases}
           I_0 ,&\ t < 0 \\
           I_s + (I_0-I_s)e^{-\frac{t}{\tau}} ,&\ 0 \leq t \\
         \end{cases}
  \end{align*}
\end{formula}
Which again shows that solving such a circuit is as simple as finding:
\begin{itemize}
  \itemsep0em
  \item $i(0)$: the initial current across the inductor
  \item $i(\infty)$: the final current across the inductor
  \item $\tau=\frac{L}{R}$: the time constant of the circuit
\end{itemize}

\subsection{Summary of First Order Circuits}
\label{ssec:summaryOfFirstOrderCircuits}

A first order circuit is one that contains a single inductor or capacitor, and is thus characterized by a first order differential equation. All the circuits in Figure \ref{fig:firstOrderCircuits} are first order circuits.

\begin{figure}[H]
  \centering
  \begin{subfigure}[H]{0.22\textwidth}
    \centering
    \includestandalone{figures/fig_032}
  \end{subfigure}
  \begin{subfigure}[H]{0.22\textwidth}
    \centering
    \includestandalone{figures/fig_033}
  \end{subfigure}
  \begin{subfigure}[H]{0.22\textwidth}
    \centering
    \includestandalone{figures/fig_034}
  \end{subfigure}
  \begin{subfigure}[H]{0.22\textwidth}
    \centering
    \includestandalone{figures/fig_035}
  \end{subfigure}
  \caption{First Order Circuits}
  \label{fig:firstOrderCircuits}
\end{figure}

\subsubsection{RC Circuits}
\label{sssec:rcCircuitsSummary}

The time constant for an RC Circuit is:
\begin{equation*}
  \tau = RC
\end{equation*}

The natural response of an RC Circuit is:
\begin{equation*}
  v(t) = V_ie^{-\frac{t}{\tau}}
\end{equation*}
and the complete response of the RC Circuit is:
\begin{align*}
  v(t) &= v(\infty) + \big[v(0)-v(\infty)\big]e^{-\frac{t}{\tau}} \\
       &= V_ie^{-\frac{t}{\tau}} + V_s\left(1-e^{-\frac{t}{\tau}}\right) \\
       &= \begin{cases}
         V_0 ,&\ t < 0 \\
         V_s + (V_0-V_s)e^{-\frac{t}{\tau}} ,&\ 0 \leq t \\
       \end{cases}
\end{align*}

\subsubsection{RL Circuits}
\label{sssec:rlCircuitsSummary}
The time constant for an RL Circuit is:
\begin{equation*}
  \tau = \frac{L}{R}
\end{equation*}

The natural response of an RL Circuit is:
\begin{equation*}
  i(t) = I_ie^{-\frac{t}{\tau}}
\end{equation*}
and the complete response of the RL Circuit is:
\begin{align*}
  i(t) &= i(\infty) + \big[i(0)-i(\infty)\big]e^{-\frac{t}{\tau}} \\
       &= I_ie^{-\frac{t}{\tau}} + I_s\left(1-e^{-\frac{t}{\tau}}\right) \\
       &= \begin{cases}
         I_0 ,&\ t < 0 \\
         I_s + (I_0-I_s)e^{-\frac{t}{\tau}} ,&\ 0 \leq t \\
       \end{cases}
\end{align*}

\end{document}
